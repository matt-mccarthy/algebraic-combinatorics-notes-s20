%!TEX root=./notes_template.tex

\noindent
{\large \bf Permutation Basics}


Comprehensive resources from an algebraic combinatorics perspective abound.  Notable examples include:

\medskip
\noindent
Macdonald's Notes on Schubert polynomials

\noindent
Garsia's ``The Saga of Reduced Factorizations of Elements of the Symmetric Group".




\begin{definition}
For $n\geq 1$, a {\it permutation} of $n$ is a 
is a bijection from  $[n]$ to $[n]$, where $[n]=\{1,2,\ldots,n\}$.
\end{definition}

\begin{example} 
A permutation of $n=3$ is defined by $\sigma(1)=2,\sigma(2)=1,\sigma(3)=3$.
\end{example}

We set some conventions to make it more convenient to express a permutation.
First, for $\sigma\in S_n$ where $\sigma(i)=j$, we instead write $\sigma_i=j$.
Two different notations to represent the action of $\sigma$ are as follows:

\bigskip
\noindent
{\it Two line notation:}
$\sigma=\left(\begin{matrix}
1&2&3&\cdots&n\\
\sigma_1&\sigma_2&\sigma_3&\cdots&\sigma_n
\end{matrix}
\right)$

\bigskip

\noindent
{\it One line notation:} $\sigma=(\sigma_1\,\sigma_2\,\ldots\,\sigma_n)$

\bigskip

\begin{example}
The permutation 
$\sigma=\left(\begin{matrix}
1&2&3&4&5\\
4&2&1&5&3
\end{matrix}
\right)
= ( 4\,2\,1\,5\,3)
$
is the permutation of 5 letters where $1\mapsto 4, 2\mapsto 2, 3\mapsto 1$ and so forth.
\end{example}

For enumerative purposes, we often think of a permutation simply as an $n$-tuple with 
distinct entries between 1 and $n$.  However, keep in mind that these are functions
and can thus be composed.
$$
\tau=
\begin{pmatrix}
1&2&3\\
3&1&2
\end{pmatrix}
\;\;\text{and} \;\;
\sigma=
\begin{pmatrix}
1&2&3\\
2&1&3
\end{pmatrix}
\implies \tau\sigma=
\begin{pmatrix}
1&2&3\\
1&3&2
\end{pmatrix}
\,.
$$
Recall that the composition of bijections is itself a bijection.  It is thus not surprising 
that the  set of all permutations forms a group.

\begin{definition}
For $n\geq 1$, the {\it symmetric group} $S_n$ is the set of permutations of $n$
under the composition of maps.  
\end{definition}
\begin{example}
The symmetric group on 3 elements, in one-line notation, is
$S_3= \left\{ (1\,2\,3),(1\,3\,2),(3\,1\,2),(2\,1\,3),(2\,3\,1),(3\,2\,1)\right\}$.
\end{example}
One line notation makes it clear that the group has order $n!$.
Note that $S_n$, for $n>2$, is not an abelian group since
the elements do not necessarily commute.
$$
\sigma\tau=
\begin{pmatrix}
1&2&3\\
3&2&1
\end{pmatrix}
$$
The {\it inverse} of $\sigma\in S_n$
is the element $\sigma^{-1}$ where
$\sigma^{-1}\sigma = id$.  Since the inverse of
a permutation $\sigma$ is the permutation that
undoes the action of $\sigma$, it can easily be 
computed from $\sigma$ by swapping the top and 
bottom row in the 2-line notation for $\sigma$.
$$
\sigma=\left(\begin{matrix}
1&2&3&\cdots&n\\
\sigma_1&\sigma_2&\sigma_3&\cdots&\sigma_n
\end{matrix}
\right)
\quad\implies\quad
\sigma^{-1}=
\left(\begin{matrix}
\sigma_1&\sigma_2&\sigma_3&\cdots&\sigma_n\\
1&2&3&\cdots&n
\end{matrix}
\right)
$$
Note that simply swapping the two rows does not yield the inverse in conventional 2-line notation but
that is easily remedied by ordering the resulting columns so that its top row is increasing.
\begin{example}
$$
\sigma = (3\,2\,5\,1\,4)\quad\implies\quad
\sigma^{-1}=
\left(\begin{matrix}
3&2&5&1&4\\
1&2&3&4&5
\end{matrix}
\right)
=
\left(\begin{matrix}
1&2&3&4&5\\
4&2&1&5&3
\end{matrix}
\right)
$$
\end{example}

\begin{definition}
A permutation that is equal to its inverse is called
an {\it involution}.  Note 
$\sigma=\sigma^{-1}\iff \sigma\sigma=id$.
\end{definition}
\begin{example}
The involutions of $S_3$ are
$\{
(1\,2\,3), (2\,1\,3), (3\,2\,1), (1\,3\,2)\}$.
\end{example}

We like to count the number of elements in a set.  Sometimes this can be done with an explicit formula
such as we did for the order of $S_n$.  However, it may be that a closed formula is hard to
come by; we then often appeal to bijections and/or recursions.  


\bigskip

\section*{\bf Homework 1}

\begin{enumerate}
\item Make sure you are comfortable with our conventions by computing $\sigma^2$ and
finding $\sigma^{-1}$ for
$$
\sigma = (6\,7\,3\,2\,5\,1\,4)\in S_7\,.
$$
Is $\sigma$ an involution?

\bigskip

\item Prove that the composition of bijections from $A\to A$ is a bijection.
\end{enumerate}









% $$
% {\tiny\tableau[scY]{ \fl,\fl,| \fl,,| ,,| }}
% \qquad
% {\tiny\tableau[scY]{\fl ,,| \fl,,| ,,| }}
% \qquad
% {\tiny\tableau[scY]{ \fl,\fl,| ,,| ,,| }}
% \qquad
% {\tiny\tableau[scY]{\fl,,| ,,| ,,| }}
% \qquad
% {\tiny\tableau[scY]{ ,,|,,| ,,| }}
% $$
% $$
% {\tiny\tableau[scY]{\bl, 8, 9, 5, 4, 6, 7 ,2, 3, 10, 1|
% \bl 1,,,,,,,,,,\times|
% \bl 2,,,,,,,\times,\circ,\circ,\circ|
% \bl 3,,,,,,,\circ,\times,\circ,\circ|
% \bl 4,,,,\times,\circ,\circ,\circ,\circ,\circ,\circ|
% \bl 5,,,\times,\circ,\circ,\circ,\circ,\circ,\circ,\circ|
% \bl 6,,,\circ,\circ,\times,\circ,\circ,\circ,\circ,\circ|
% \bl 7,,,\circ,\circ,\circ,\times,\circ,\circ,\circ,\circ|
% \bl 8,\times,\circ,\circ,\circ,\circ,\circ,\circ,\circ,\circ,\circ|
% \bl 9,\circ,\times,\circ,\circ,\circ,\circ,\circ,\circ,\circ,\circ|
% \bl 10,\circ,\circ,\circ,\circ,\circ,\circ,\circ,\circ,\times,\circ|
% }}
% $$
